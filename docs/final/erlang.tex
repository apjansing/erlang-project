% template for papers with a title page
% see dgstpp.sty for title page info
% format: latex
% last changed: 1 Apr 2015

\documentclass[11 pt]{IEEEtran}

% standard math packages
\usepackage{amsmath,amsfonts,amssymb}
\usepackage{scrextend}
\usepackage{tikz}
\usetikzlibrary{arrows, shapes, plothandlers}
\usepackage{tikz}
\usepackage{tikzscale}
\usepackage{ragged2e}
\usepackage[nolabel, final]{showlabels}

\usepackage{hyperref}
\usepackage{url}
\makeatletter
\g@addto@macro{\UrlBreaks}{\UrlOrds}
\makeatother


\usepackage{listings}
\usepackage{color}

\definecolor{dkgreen}{rgb}{0,0.6,0}
\definecolor{gray}{rgb}{0.5,0.5,0.5}
\definecolor{mauve}{rgb}{0.58,0,0.82}

\lstset{frame=tb,
  language=erlang,
  aboveskip=1mm,
  belowskip=1mm,
  showstringspaces=false,
  columns=flexible,
  basicstyle={\scriptsize\ttfamily},
  numbers=none,
  numberstyle=\tiny\color{gray},
  keywordstyle=\color{blue},
  commentstyle=\color{dkgreen},
  stringstyle=\color{mauve},
  breakatwhitespace=true,
  tabsize=2,
  breaklines=true,
  literate={<div>}{<div>}{1\discretionary{}{}{}}{</div>}{</div>}{1\discretionary{}{}{}}
  {N}{N}{1\discretionary{}{}{}}{removeAttribute}{removeAttribute}{1\discretionary{}{}{}}{include}{include}{1\discretionary{}{}{}},
  postbreak=\mbox{\textcolor{red}{$\hookrightarrow$}\space}
}


% Phil Parker's DGS packages, some modified
\usepackage{remexpp,pprroof,dgstpp}

% other packages
\usepackage{setspace}
%\usepackage{hyperref,color}

% "fancy" font
\usepackage{fourier}
\usepackage[T1]{fontenc}
   
% make reference header the right font size
\renewcommand\refname{\Large References}
   
% theorems, remarks, etc using Phil Parker's "remexpp.sty"
\newtheorem{theorem}{Theorem}[section]
\newtheorem{prop}[theorem]{Proposition}
\newtheorem{lemma}[theorem]{Lemma}
\newtheorem{claim}[theorem]{Claim}
\newtheorem{corollary}[theorem]{Corollary}
\newremark{definition}[theorem]{Definition}
\newremark{example}[theorem]{Example}
\newremark{remark}[theorem]{Remark} 

% make rsfs, TeX \cal, and Euler script *all* available
\usepackage{mathrsfs}
\let\rscr=\mathscr
\let\mathscr=\relax
\let\mcal=\mathcal
\usepackage{eucal}
\let\escr=\mathcal
\let\mathcal=\relax

% commutative diagrams with XY-pic
\usepackage[all]{xy}
\SelectTips{cm}{}

\arraycolsep .2em
   
% new commands
\renewcommand{\a}{\alpha}
\newcommand{\Aut}[1]{\textrm{Aut}(#1)}
\newcommand{\B}{\rscr{B}}
\newcommand{\br}[2]{\left[#1,#2\right]}
\newcommand{\bre}{\br{\ }{\,}}
\newcommand{\ddg}{\ddot{\g}}
\newcommand{\dg}{\dot{\g}}
\newcommand{\DGS}{D{\kern-.375em}G{\kern-.2em}S}
\newcommand{\ds}{\oplus}
\newcommand{\eB}{\escr{B}}
\newcommand{\eH}{\escr{H}}
\newcommand{\eI}{\escr{I}}
\newcommand{\eV}{\escr{V}}
\newcommand{\g}{\gamma}
\newcommand{\G}{\Gamma}
\newcommand{\h}{\lal{h}}
\renewcommand{\H}{\rscr{H}}
\newcommand{\hp}{\h_{2p + 1}}
\newcommand{\iso}{\cong}
\newcommand{\lag}{\mathfrak{g}}
\newcommand{\lal}[1]{\mathfrak{#1}}
\newcommand{\n}{\lal{n}}
\newcommand{\pplus}{+\mspace{-10 mu}+}
\newcommand{\R}{\mathbb{R}}
\newcommand{\rS}{\rscr{S}}
\renewcommand{\span}[1]{[\mspace{-3.25 mu}[ #1 ]\mspace{-3.25 mu}]}
\newcommand{\surj}{\rightarrow\kern-.82em\rightarrow}
\newcommand{\tQ}{\widetilde{Q}}
\renewcommand{\v}{\lal{v}}
\newcommand{\V}{\rscr{V}}
\newcommand{\z}{\lal{z}}
%%Alex's defined commands%%
\newcommand{\adx}{ad$_x$ }
\newcommand{\ady}{ad$_y$ }
\newcommand{\adz}{ad$_z$ }
\newcommand{\fj}{\mathfrak{j}}
\newcommand{\fg}{\mathfrak{g}}
\newcommand{\fz}{\mathfrak{z}}
\newcommand{\fv}{\mathfrak{v}}
\newcommand{\fh}{\mathfrak{h}}
\newcommand{\QQ}{\mathbb{Q}}
\newcommand{\ZZ}{\mathbb{Z}}
\newcommand{\RR}{\mathbb{R}}
\newcommand{\CC}{\mathbb{C}}
\newcommand{\NN}{\mathbb{N}}
\newcommand{\FF}{\mathbb{F}}


\makeatletter
\newcommand{\ad}[1]{\mathop{\operator@font ad}\nolimits_{#1}}
\makeatother

% show labels in margin (must be last package added)
\usepackage{showlabels}

% input information for the title page here:
\preprint{}
\title{Programming Languages: Erlang}
\author{Alexander Jansing}
\address{
   Computer Science Department\\
   State University of New York,\\
   Polytechnic Institute\\
   Utica, NY 13502\\
   USA\\
   \textsf{jansina@sunypoly.edu}
}
\date{\today}
\abstract{
Joe Armstrong, one of Erlang's original authors, analogized the language to Java's ``write once, run anywhere,'' to ``write once, run forever''\cite{run-forever}. Erlang is a functional programming language that is known for it use in the telecom industry and message queuing and devops software like \emph{RabbitMQ}\cite{ribbitmq-wiki} and \emph{Chef}\cite{chef-wiki}, respectively. Erlang is sometime referred to as Erlang/OTP or just OTP (Open Telecom Platform).
}
\msc{}{}

\begin{document}
\maketitle


\section{Introduction}
\subsection{History of Erlang}
Erlang was originally developed within Ericsson by Joe Armstrong, Robert Virding, and Mike Williams in 1986 and released to open source in 1998. Erlang is sometime referred to as Erlang/OTP or just OTP (Open Telecom Platform). It is still maintained by the OTP product unit at Ericsson \cite{wiki}. 

Erlang was designed with the purpose of improving the development telephony applications. As recently as 2014, Erlang was used in the support nodes of 3G and LTE mobile network nodes\cite{wiki}\cite{inside-erlang}.

\subsection{Layout of Paper}
This report will cover:
\begin{enumerate}
    \item the research I performed on the language,
    \item Erlang's use cases,
    \item and what I have been to do with Erlang.
\end{enumerate}

If any definitions are missing, you may find it in the Appendix \ref{definitions}.

\section{Research}
The primary way I learn new languages is to find a variety of tutorials and many quick searches to figure out what I need to do when tasks come up. In addition to on-demand searches, use cases are of great importance to the research and decision whether-or-not to use a language for a given task. Use cases will be covered more in the Section \ref{usecase}.

Erlang can be used for variety of tasks, but has never gained much popularity outside of its original intent. Erlang seems to have an artificially high learning curve. Most sources I found assumed a solid foundation of knowledge in the language and just provided blocks of code with limited explanation of why things were being done that way they were. It reminds me of the web developement community that assumes you know what they are talking about (\emph{i.e.} using React, Redux, jQuery, etc.) without explaining what modules or components are or even provide sources to aid those who do not already know.

The tutorials that I found most useful were on Tutorialspoint\cite{tutorialspoint} and O'Reilly\cite{kessin}. Even though these tutorials were the most useful, they were still some of the drier tutorials I have followed. I will cover more on these tutorials in Subsection \ref{tutorials}.

\section{Erlang Use Cases}\label{usecase}
Joe Armstrong, one of Erlang's original authors, analogized the language to Java's ``write once, run anywhere,'' to ``write once, run forever''\cite{run-forever}. Erlang is a functional programming language that is known for it use in the telecommunications industry and message queuing and devops software like \emph{RabbitMQ}\cite{ribbitmq-wiki} and \emph{Chef}\cite{chef-wiki}, respectively. 

Erlang has several characteristics that make it very attractive to those who need their programs to stay alive, like the telecommunications industry. It supports \emph{distribution}, \emph{hot swapping}, and \emph{soft real-time computing}.

\section{Project}\label{project}
\subsection{Learning}
\subsubsection{Tutorials}\label{tutorials}
Earlier I mentioned that that two tutorials that I found most useful were on Tutorialspoint\cite{tutorialspoint} and O'Reilly\cite{kessin}. The O'Reilly tutorial, authored by Zachary Kessin, does a great job explaining how to interact with maps (\emph{i.e.} JSON) and specified schemas that descripe the expected input to a program. This tutorial was the first attempt to create a web app with erlang that I could find, but it would have required diving deeper into learning CoffeeScript and the ExtJS framework\cite{sencha}. For the sake of staying on task, staying within the scope of this assignment, and the time bounds of the course I searched for and worked witht eh Tutorialspoint web app example.

The Tutorialspoint web app tutorial was a fairly straightforward example of how to create a web server and delivier some code over a port. The tutorial can run as provided, but it helps to have a better understanding of \emph{httpd}

\subsubsection{Using Erlang}
Erlang can be a powerful language, but it seems that there will be quite a learning curve on how to use the language. I installed the Erlang, written a couple small programs (see Appendix \ref{code-snippets}), but to perform more complex tasks you would need to use \emph{rebar3}. Rebar3 is an application and dependency management program for erlang. This would be analogous with other dependency management tools for other languages like \emph{Maven} (Java) and \emph{pip} (Python).

\subsection{My Erlang Programs}\label{programs}
\subsubsection{The Basics}
The first three programs that I wrote in erlang were:
\begin{enumerate}
    \item hello\_world.erl (\ref{hw})
    \item basic\_math.erl (\ref{bm})
    \item and fibonacci.erl (\ref{fib})
\end{enumerate}
These three programs were exercises in showing off the basic syntax of erlang and how it can perform actions similar to OCaml or Haskel. 

$(1)$ The hello\_world.erl program is the basic program everyone writes up when learning pretty much anything new in programming. It shows how a module is declared, how to export a function for later use, and how to print out a basic string. $(2)$ The basic\_math.erl program continues on from where hello\_world.erl left off by showing how to import a function from a module, export function with parameters, some basic math operations, append to a list, and print a list using \textsf{io:fwrite}. $(3)$ And finally, fibonacci.erl shows how erlang performs pattern matching, conditionals, and recursion.

\subsubsection{Diving Deeper}
After experimenting with the basics of erlang, I wanted to see what else I could do with a langauge that started out as a proprietary language for telecommunications in a local environment. My first thought was to create some kind of chat application that could sent messages between two docker containers or two computers on a local network, but after some research all I was able to complete was sending messages between two users logged into localhost on the same computer. This was performed by using the erlang module called \emph{erlbus}\cite{erlbus}. Erlbus taught me how to use rebar3, but the results of my experiments did not result in much beyond what was described in the README.

During my research I found the Tutorialspoint and O'Reilly tutorials that show how to create a webserver that could host web apps. The example I provide in the Appendix \ref{pa} serves up my github portfolio page (\url{https://apjansing.github.io}) at \url{http://http://localhost:8081/erl/poly_app:service} after running the following commands:
\begin{lstlisting}
git clone git@github.com:apjansing/erlang-project.git
cd erlang-project/examples/web_server
erlc poly_app.erl
erl
> c(poly_app).
> inets:start().
> poly_app:start().
\end{lstlisting}
This web server is interesting, because it has lead to the reverse proxying program YAWS\cite{yaws} \`a-la Apache HTTP Server Project \cite{apache-httpd}.

\section{Conclusion}
Erlang is a powerful language that has never won any popularity contents or is very fast. It is known to be resilient and tolerant of hot-swapping code\cite{wiki}. Erlang has a steep learning curve. The erlang community has many people who love their language and want to help each other. But the questions pertaining to basic questions are limited. If you have the time, reading Fred H\`erbert's book might be the best way to learn erlang\cite{herbert}. I used his book while learning some of the basic syntax, but he is very thourough and better for someone who is focusing all of their attention on absorbing what he has to say.

% \newpage
\section{Appendix}
\subsection{Definitions}\label{definitions}
\begin{enumerate}
\item  Hot Swapping - code can be updated without shutting down a system \cite{wiki}.
\item Soft Real-time Computing - computation where the goal is to meet subsets of deadlines in order to optimize application-specific criteria (defined by developer). \cite{soft-rt}.
\end{enumerate}
 
\subsection{Code snippets}\label{code-snippets}

    \subsubsection{hello\_world.erl} \label{hw}\hfill
    \lstinputlisting{../../examples/hello_world/hello_world.erl}
    
    \subsubsection{basic\_math.erl} \label{bm}\hfill
    \lstinputlisting{../../examples/basic_math/basic_math.erl}

    \subsubsection{fibonacci.erl} \label{fib}\hfill
    \lstinputlisting{../../examples/fibonacci/fibonacci.erl}
    
    \subsubsection{poly\_app.erl} \label{pa}\hfill
    \lstinputlisting{../../examples/web_server/poly_app.erl}
\subsection{Source locations}

  
\begin{thebibliography}{99}
\bibitem{erlang}
Erlang.org. (2019). Erlang Programming Language. [online] Available at: \url{http://www.erlang.org/} [Accessed 19 Mar. 2019].

\bibitem{0xAX}
GitHub. (2019). 0xAX/erlang-bookmarks. [online] Available at: \url{https://github.com/0xAX/erlang-bookmarks} [Accessed 19 Mar. 2019].

\bibitem{Carrone}
Carrone, F. (2019). Become an Erlang Cowboy and tame the Wild Wild West -- Part I. [online] This is not a Monad tutorial. Available at: \url{https://notamonadtutorial.com/become-an-erlang-cowboy-and-tame-the-wild-wild-web-part-i-37f8dd1df160} [Accessed 19 Mar. 2019].

\bibitem{Hebert}
Hebert, F. (2019). On Erlang's Syntax. [online] Ferd.ca. Available at: \url{https://ferd.ca/on-erlang-s-syntax.html} [Accessed 19 Mar. 2019].

\bibitem{Miller}
Miller, E. (2019). Why I Program in Erlang – Evan Miller. [online] Evanmiller.org. Available at: https://www.evanmiller.org/why-i-program-in-erlang.html [Accessed 19 Mar. 2019].

\bibitem{Notepad}
Quick Notepad Tutorial (2019). Erlang Hello World Example How To Write Compile and Execute Erlang Program on Linux. [online] YouTube. Available at: \url{https://www.youtube.com/watch?v=zGG9eI13UHA} [Accessed 20 Mar. 2019].

\bibitem{wiki}
Wikiwand. (2019). Erlang (programming language) | Wikiwand. [online] Available at: \url{https://www.wikiwand.com/en/Erlang_(programming_language)} [Accessed 20 Mar. 2019].

\bibitem{ribbitmq-wiki}
Wikiwand. (2019). RabbitMQ | Wikiwand. [online] Available at: \url{https://www.wikiwand.com/en/RabbitMQ} [Accessed 20 Mar. 2019].

\bibitem{chef-wiki}
Wikiwand. (2019). Chef (software) | Wikiwand. [online] Available at: \url{https://www.wikiwand.com/en/Chef_(software)} [Accessed 20 Mar. 2019].

\bibitem{run-forever}
YouTube. (2019). Rackspace takes a look at the ERLANG programming language for distributed computing. [online] Available at: \url{https://www.youtube.com/watch?v=u41GEwIq2mE&t=3m59s} [Accessed 20 Mar. 2019].

\bibitem{soft-rt}
“Real-time computing,” Wikiwand. [Online]. Available: \url{https://www.wikiwand.com/en/Real-time_computing#/Soft}. [Accessed: 21-Mar-2019].

\bibitem{inside-erlang}
“Inside Erlang – creator Joe Armstrong tells his story,” Ericsson.com, 04-Dec-2014. [Online]. Available: \url{https://www.ericsson.com/en/news/2014/12/inside-erlang--creator-joe-armstrong-tells-his-story}. [Accessed: 21-Mar-2019].

\bibitem{erlbus}
McCord, C. and Bolaños, C. (2019). cabol/erlbus. [online] GitHub. Available at: https://github.com/cabol/erlbus [Accessed 27 Apr. 2019].

\bibitem{gsmith}
G. Smith, YouTube - Building a web app in Erlang - yes you heard me right I said Erlang not Elixir, 25-Mar-2017. [Online]. Available: \url{https://www.youtube.com/watch?v=BO-8Hx8kPtA}. [Accessed: 03-May-2019].

\bibitem{tutorialspoint}
Tutorialspoint.com, ``Erlang Web Programming,'' www.tutorialspoint.com. [Online]. Available: \url{https://www.tutorialspoint.com/erlang/erlang_web_programming.htm}. [Accessed: 03-May-2019].

\bibitem{kessin}
Kessin, Z. (2019). Building Web Applications with Erlang. [online] O'Reilly | Safari. Available at: \url{https://www.oreilly.com/library/view/building-web-applications/9781449320621/ch04.html#rest} [Accessed 5 May 2019].

\bibitem{yaws}
GitHub. (2019). klacke/yaws. [online] Available at: https://github.com/klacke/yaws [Accessed 6 May 2019].

\bibitem{inets}
Erldocs.com. (2019). inets (inets) - (Erlang Documentation). [online] Available at: \url{https://erldocs.com/r15b01/inets/inets.html} [Accessed 6 May 2019].

\bibitem{modesi}
Erlang.org. (2019). Erlang -- mod\_esi. [online] Available at: \url{http://erlang.org/doc/man/mod_esi.html} [Accessed 6 May 2019].

\bibitem{sencha}
Sencha.com. (2019). [online] Available at: \url{http://sencha.com/} [Accessed 6 May 2019].

\bibitem{apache-httpd}
Httpd.apache.org. (2019). Welcome! - The Apache HTTP Server Project. [online] Available at: https://httpd.apache.org/ [Accessed 7 May 2019].

\bibitem{herbert}
Hérbert, F. (2019). Learn You Some Erlang for Great Good!. [online] Learnyousomeerlang.com. Available at: https://learnyousomeerlang.com/ [Accessed 7 May 2019].

\end{thebibliography}
\end{document}






















