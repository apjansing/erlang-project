% template for papers with a title page
% see dgstpp.sty for title page info
% format: latex
% last changed: 1 Apr 2015

\documentclass[11 pt]{IEEEtran}

% standard math packages
\usepackage{amsmath,amsfonts,amssymb}
\usepackage{scrextend}
\usepackage{tikz}
\usetikzlibrary{arrows, shapes, plothandlers}
\usepackage{tikz}
\usepackage{tikzscale}
\usepackage{ragged2e}

\usepackage{hyperref}
\usepackage{url}
\makeatletter
\g@addto@macro{\UrlBreaks}{\UrlOrds}
\makeatother


\usepackage{listings}
\usepackage{color}

\definecolor{dkgreen}{rgb}{0,0.6,0}
\definecolor{gray}{rgb}{0.5,0.5,0.5}
\definecolor{mauve}{rgb}{0.58,0,0.82}

\lstset{frame=tb,
  language=bash,
  aboveskip=3mm,
  belowskip=3mm,
  showstringspaces=false,
  columns=flexible,
  basicstyle={\small\ttfamily},
  numbers=none,
  numberstyle=\tiny\color{gray},
  keywordstyle=\color{blue},
  commentstyle=\color{dkgreen},
  stringstyle=\color{mauve},
  breakatwhitespace=true,
  tabsize=3,
  breaklines=true,
  postbreak=\mbox{\textcolor{red}{$\hookrightarrow$}\space}
}


% Phil Parker's DGS packages, some modified
\usepackage{remexpp,pprroof,dgstpp}

% other packages
\usepackage{setspace}
%\usepackage{hyperref,color}

% "fancy" font
\usepackage{fourier}
\usepackage[T1]{fontenc}
   
% make reference header the right font size
\renewcommand\refname{\Large References}
   
% theorems, remarks, etc using Phil Parker's "remexpp.sty"
\newtheorem{theorem}{Theorem}[section]
\newtheorem{prop}[theorem]{Proposition}
\newtheorem{lemma}[theorem]{Lemma}
\newtheorem{claim}[theorem]{Claim}
\newtheorem{corollary}[theorem]{Corollary}
\newremark{definition}[theorem]{Definition}
\newremark{example}[theorem]{Example}
\newremark{remark}[theorem]{Remark} 

% make rsfs, TeX \cal, and Euler script *all* available
\usepackage{mathrsfs}
\let\rscr=\mathscr
\let\mathscr=\relax
\let\mcal=\mathcal
\usepackage{eucal}
\let\escr=\mathcal
\let\mathcal=\relax

% commutative diagrams with XY-pic
\usepackage[all]{xy}
\SelectTips{cm}{}

\arraycolsep .2em
   
% new commands
\renewcommand{\a}{\alpha}
\newcommand{\Aut}[1]{\textrm{Aut}(#1)}
\newcommand{\B}{\rscr{B}}
\newcommand{\br}[2]{\left[#1,#2\right]}
\newcommand{\bre}{\br{\ }{\,}}
\newcommand{\ddg}{\ddot{\g}}
\newcommand{\dg}{\dot{\g}}
\newcommand{\DGS}{D{\kern-.375em}G{\kern-.2em}S}
\newcommand{\ds}{\oplus}
\newcommand{\eB}{\escr{B}}
\newcommand{\eH}{\escr{H}}
\newcommand{\eI}{\escr{I}}
\newcommand{\eV}{\escr{V}}
\newcommand{\g}{\gamma}
\newcommand{\G}{\Gamma}
\newcommand{\h}{\lal{h}}
\renewcommand{\H}{\rscr{H}}
\newcommand{\hp}{\h_{2p + 1}}
\newcommand{\iso}{\cong}
\newcommand{\lag}{\mathfrak{g}}
\newcommand{\lal}[1]{\mathfrak{#1}}
\newcommand{\n}{\lal{n}}
\newcommand{\pplus}{+\mspace{-10 mu}+}
\newcommand{\R}{\mathbb{R}}
\newcommand{\rS}{\rscr{S}}
\renewcommand{\span}[1]{[\mspace{-3.25 mu}[ #1 ]\mspace{-3.25 mu}]}
\newcommand{\surj}{\rightarrow\kern-.82em\rightarrow}
\newcommand{\tQ}{\widetilde{Q}}
\renewcommand{\v}{\lal{v}}
\newcommand{\V}{\rscr{V}}
\newcommand{\z}{\lal{z}}
%%Alex's defined commands%%
\newcommand{\adx}{ad$_x$ }
\newcommand{\ady}{ad$_y$ }
\newcommand{\adz}{ad$_z$ }
\newcommand{\fj}{\mathfrak{j}}
\newcommand{\fg}{\mathfrak{g}}
\newcommand{\fz}{\mathfrak{z}}
\newcommand{\fv}{\mathfrak{v}}
\newcommand{\fh}{\mathfrak{h}}
\newcommand{\QQ}{\mathbb{Q}}
\newcommand{\ZZ}{\mathbb{Z}}
\newcommand{\RR}{\mathbb{R}}
\newcommand{\CC}{\mathbb{C}}
\newcommand{\NN}{\mathbb{N}}
\newcommand{\FF}{\mathbb{F}}


\makeatletter
\newcommand{\ad}[1]{\mathop{\operator@font ad}\nolimits_{#1}}
\makeatother

% show labels in margin (must be last package added)
\usepackage{showlabels}

% input information for the title page here:
\preprint{}
\title{Programming Languages: Erlang}
\author{Alexander Jansing}
\address{
   Computer Science Department\\
   State University of New York,\\
   Polytechnic Institute\\
   Utica, NY 13502\\
   USA\\
   \textsf{jansina@sunypoly.edu}
}
\date{\today}
\abstract{
Erlang is a functional programming language that is known for it use in the telecom industry and the popular software like \emph{RabbitMQ}\cite{ribbitmq-wiki} and \emph{Chef}\cite{chef-wiki}. Erlang is sometime referred to as Erlang/OTP or just OTP (Open Telecom Platform). Joe Armstrong Erlang's original author, analogized the language to Java's ``write once, run anywhere,' to ``'write once, run forever.''

%Apache Apex is Hadoop YARN-native framework for building distributed applications and applies native streaming to the data processing pipeline \cite{WEISE}. The Apex project was mainly been driven by the company DataTorrent. DataTorrent shut its doors back in May of 2018\cite{WIKI}\cite{DATANAMI}.
}
\msc{}{}

\begin{document}
\maketitle


\section{Introduction}
\subsection{History of Erlang}
\subsection{Erlang Use Cases}
 

\section{Research}
 
\section{Using Apex}

\section{Stretch Goals and Future Work}


\newpage
\section{Appendix}
 
\subsection{Code snippets}

%\begin{lstlisting}
%
%\end{lstlisting}


 \subsection{Source locations}

  
\begin{thebibliography}{99}
\bibitem{erlang}
Erlang.org. (2019). Erlang Programming Language. [online] Available at: \url{http://www.erlang.org/} [Accessed 19 Mar. 2019].

\bibitem{0xAX}
GitHub. (2019). 0xAX/erlang-bookmarks. [online] Available at: \url{https://github.com/0xAX/erlang-bookmarks} [Accessed 19 Mar. 2019].

\bibitem{Carrone}
Carrone, F. (2019). Become an Erlang Cowboy and tame the Wild Wild West -- Part I. [online] This is not a Monad tutorial. Available at: \url{https://notamonadtutorial.com/become-an-erlang-cowboy-and-tame-the-wild-wild-web-part-i-37f8dd1df160} [Accessed 19 Mar. 2019].

\bibitem{Hebert}
Hebert, F. (2019). On Erlang's Syntax. [online] Ferd.ca. Available at: \url{https://ferd.ca/on-erlang-s-syntax.html} [Accessed 19 Mar. 2019].

\bibitem{Miller}
Miller, E. (2019). Why I Program in Erlang – Evan Miller. [online] Evanmiller.org. Available at: https://www.evanmiller.org/why-i-program-in-erlang.html [Accessed 19 Mar. 2019].

\bibitem{Notepad}
Quick Notepad Tutorial (2019). Erlang Hello World Example How To Write Compile and Execute Erlang Program on Linux. [online] YouTube. Available at: \url{https://www.youtube.com/watch?v=zGG9eI13UHA} [Accessed 20 Mar. 2019].

\bibitem{wiki}
Wikiwand. (2019). Erlang (programming language) | Wikiwand. [online] Available at: \url{https://www.wikiwand.com/en/Erlang_(programming_language)} [Accessed 20 Mar. 2019].

\bibitem{ribbitmq-wiki}
Wikiwand. (2019). RabbitMQ | Wikiwand. [online] Available at: \url{https://www.wikiwand.com/en/RabbitMQ} [Accessed 20 Mar. 2019].

\bibitem{chef-wiki}
Wikiwand. (2019). Chef (software) | Wikiwand. [online] Available at: \url{https://www.wikiwand.com/en/Chef_(software)} [Accessed 20 Mar. 2019].

\bibitem{run-forever}
YouTube. (2019). Rackspace takes a look at the ERLANG programming language for distributed computing. [online] Available at: \url{https://www.youtube.com/watch?v=u41GEwIq2mE&t=3m59s} [Accessed 20 Mar. 2019].

\end{thebibliography}
\end{document}






















